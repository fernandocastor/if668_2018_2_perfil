\documentclass[a4paper]{article}
\usepackage[brazil]{babel}
\usepackage[utf8]{inputenc}
\fontsize{10}{10}
\usepackage[a4paper,top=3cm,bottom=2cm,left=2cm,right=2cm]{geometry}

\title{IF781 - Empreendimentos em Informática}
\author{Thomas Patrick Gomes de Alcântara }
\date{15 de Outubro de 2018}

\usepackage{natbib}
\usepackage{graphicx}

\begin{document}

\maketitle

\section{Introdução}
Empreendimentos em Informática(IF781) é uma disciplina eletiva do CIN que é ofertada para os cursos de Ciência e Engenharia da Computação como eletiva do Perfil Empreendedor dos cursos. Essa disciplina tem como objetivo estimular a criação de novas empresas na área de informática começando de dentro da Universidade, usando conhecimentos técnicos em computação e em áreas administrativas.


\section{Relevância}
Como uma disciplina optativa da área de empreendedorismo, essa disciplina é muito importante para quem deseja começar seu projeto de negócios na área de informática, esse curso dispõe de vários materiais desde livros à artigos acadêmicos como referência para consultas, por exemplo o Manual do empreendedorismo \cite{Manual} e o livro sobre Planejamento Estratégico \cite{Planejamento} que são duas leituras importantes para o aluno ter melhores noções sobre o mercado e sobre como sobreviver a ele e além dessas referências o professor também recomenda vários artigos e, entre eles um sobre Processos Criativos e o modo como eles servem para a inovação.\cite{Inovacao} E por essa atenção e estímulo à inovação, o curso está sempre em parceria com empresas que têm programas de auxílio aos alunos que querem empreender, como exemplo disso temos a SOFTEX que estava presente desde o primeiro semestre da disciplina, e mais atualmente o CESAR, que vem acolhendo os projetos que serão transformados em empresas reais. 

\section{Relação com outras disciplinas}
\begin{table}[h!]
\centering
\label{tabela} Disciplinas relacionadas à Empreendimentos em Informática.
\begin{tabular}{|l|l|}
\hline
Projeto de Desenvolvimento(IF683) & \begin{tabular}[c]{@{}l@{}}Essa disciplina do 5º período é onde o aluno começa a entender os conceitos\\ e ferramentas utilizadas em uma empresa real. É onde começa a se formar o\\ perfil profissional do estudante e nessa disciplina pede-se que seja desenvolvido\\ um projeto da maneira que se faz numa empresa real, o que leva muitas\\ vezes  o aluno a levar o projeto adiante e virar um empreendimento real, o\\ que faz a disciplina se relacionar com a de empreendimentos.
\end{tabular} 
\\ \hline
\begin{tabular}[c]{@{}l@{}}\\ Economia para\\ Empreendedores(IF780)\end{tabular} & 
\begin{tabular}[c]{@{}l@{}}Essa disciplina introduz vários conceitos econômicos e estratégicos de natureza \\ técnica para a administração de uma empresa, apresenta alguns modelos\\ estatísticos e de análise de mercado e concorrência, de modo que se\\ relaciona com a disciplina de empreendimentos para que o aluno tenha noções\\ técnicas para gerir a empresa.
\end{tabular} \\ \hline

\begin{tabular}[c]{@{}l@{}}\\ Gestão de Negócios(IF783)\end{tabular}          & \begin{tabular}[c]{@{}l@{}}Esta disciplina é de suma importância no perfil empreendedor, pois ela\\ apresenta todos os conceitos de gestão dentro de uma empresa, o aluno\\ aprenderá gestão de pessoas, financeira, de produção, marketing, além  dos \\ aspectos jurídicos de uma empresa, informações essenciais para quem deseja\\ empreender e mais ainda para quem deseja ser gestor de uma empresa.\end{tabular}                                      
\\ \hline
\begin{tabular}[c]{@{}l@{}}Contabilidade Financeira \\ e Gerencial(IF784)\end{tabular} & 
\begin{tabular}[c]{@{}l@{}} Segundo o site da disciplina, ela se propõe a complementar o aprendizado do\\ ciclo empreendedor iniciado em Empreendimentos em Informática, ensinando\\ conceitos financeiros, tributários e de análise financeira para a empresa, \\complementando assim o básico que o aluno precisa saber para empreender\\ com sucesso.
\end{tabular}                                                 \\ \hline
\end{tabular}
\end{table}

\section{Conclusão}
Atualmente, onde tudo acontece muito rápido, a possibilidade de começar seu próprio negócio ainda na Universidade é muito importante e benéfica para o aluno e para o mercado, que desse modo podem estar sempre na vanguarda com novos projetos inovadores e profissionais com ideias "frescas" chegando ao mercado.
\newline Essa disciplina, em conjunto com as outras do ciclo empreendedor do curso são de suma importância para o aluno que deseja começar a desenvolver seu projeto ainda na universidade e seguir em frente com ele à medida que aprende novos conceitos e habilidades para se desenvolver no mercado como um empreendedor ou saber como o mercado funciona para se inserir em uma empresa. 
\bibliographystyle{plain}
\bibliography{tpga}
\end{document}
