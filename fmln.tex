\documentclass{article}
\usepackage[utf8]{inputenc}
\usepackage{booktabs}

\title{IF674 - Infra-Estrutura de Hardware}
\author{Francisco Marcos Lira do Nascimento}
\date{October 2018}

\usepackage{natbib}
\usepackage[brazil]{babel}
\usepackage[utf8]{inputenc}
\usepackage{graphicx}

\begin{document}

\maketitle

\begin{figure}[h!]
  \centering
  \includegraphics[width=.5\textwidth]{c}
  \caption{}\cite{c}
\end{figure}

\section{Introduction}
O dicipliana Infra-Estrutura de Hardware tem como objetivo prover uma visão geral dos componentes do computador como processador, sistema de memória(memória principal e memória cache),Entrada e Saída e Barramentos\citep{a}

\section{Relevância}
Além dos conceitos básicos, serão apresentados conceitos avançados como pipeline e super-escalares, técnicas usadas nos processadores comerciais atuais e que garantem um grande aumento no desempenho da máquina. Neste curso o aluno conhecerá as principais tecnologias de memória e o princípio de funcionamento de cada uma delas.

\section{Bibliografia}
\begin{enumerate}
\item Computer Organization and Design RISC-V Edition - The Hardware Software Interface - David Patterson e John Hennessy\cite{b}
\item Arquitetura e Organização de Computadores: William Stallings \cite{d}

\end{enumerate}

\section{}

\begin{table}[h]
 \centering
% distancia entre a linha e o texto
 {\renewcommand\arraystretch{1.25}
 \caption{A sample table}
 \begin{tabular}{ 1 1  }
  \cline{1-1}\cline{2-2}  
    \multicolumn{1}{|p{3.850cm}|}{\cellcolor{}Column 1 \centering } &
    \multicolumn{1}{p{8.217cm}|}{\cellcolor{}Column 2 \centering }
  \\  
  \cline{1-1}\cline{2-2}  
    \multicolumn{1}{|p{3.850cm}|}{IF675 - Sistemas Digitais} &
    \multicolumn{1}{p{8.217cm}|}{O curso de Sistemas Digitais visa dar ao aluno conhecimentos de circuitos lógicos digitais combinacionais e seqüenciais cobrindo desde dispositivos digitais de pequena complexidade SSI, até a implementação de circuitos de média complexidade MSI}
  \\  
  \cline{1-1}\cline{2-2}  
    \multicolumn{1}{|p{3.850cm}|}{IF677 - Infra-Estrutura de Software \raggedleft } &
    \multicolumn{1}{p{8.217cm}|}{  Este curso faz parte da tríade hardware, software e comunicação, que é a base da construção de praticamente qualquer sistema de computação atual. O objetivo aqui é apresentar os conceitos e sistemas de software básicos de um computador, que compreende a introdução aos sistemas concorrentes e aos sistemas operacionais, sejam eles mono-computador ou distribuídos.}
  \\  
  \hline

 \end{tabular} }
\end{table}


\bibliographystyle{plain}
\bibliography{references}



\end{document}
