\documentclass[10pt, a4paper]{article}
\usepackage[utf8]{inputenc}

\usepackage[brazil]{babel}
\usepackage[utf8]{inputenc}
\usepackage[T1]{fontenc}

\title{IF699 - Aprendizagem De Máquina}
\author{Victor Gaudiot}
%\date


\usepackage{natbib}
\usepackage{graphicx}

\begin{document}

\maketitle

\begin{figure}[h!]
\centering
\includegraphics[scale=0.25]{machine-learning.png}
\caption{Aprendizagem De Máquina}
\label{fig:cérebro de máquina}
\end{figure}

\section{Introdução}
Ofertada  no primeiro semestre de cada ano, é uma cadeira do 6º período do curso do tipo eletiva, e tendo como professor George Darminton.

Ela se insere na área de inteligência artificial. O objetivo da disciplina é o estudo de métodos e algoritmos que obtém conhecimento ao analisar bases de dados.\citep{CinWikiDisciplina} 

Aborda assuntos tais como: Árvore de decisão, distâncias heterogêneas, avaliação de hipóteses, dentre outros.\citep{siteDisciplina}

\section{Relevância}
O interesse nesta área decorre no aumento da acessibilidade de informações, do custo/benefício dos processadores, armazenamento de dados e etc.\citep{Site1}

Nesta perspectiva, conclui-se que há a possibilidade da produção rápida e automática de modelos com capacidade de verificar dados maiores e mais complexos, e entregar resultados com maior rapidez e precisão, mesmo numa escala grande.

\section{Relação com outras disciplinas}
\begin{table}
\centering
\begin{tabular}{|l|l|} 
\hline
IF702 - Redes Neurais                                                              & \begin{tabular}[c]{@{}l@{}}Se relacionam por ter estruturas que promovem a \\classificação de bases de dados através do treinamento prévio.\end{tabular}  \\ 
\hline
\begin{tabular}[c]{@{}l@{}}IF685 - Gerenciamento Dados\\~e Informação\end{tabular} & Se relacionam pelo fato do intenso manuseio em banco de dados                                                                                                                                 \\
\hline
\end{tabular}
\end{table}


\bibliographystyle{plain}
\bibliography{references}


\end{document}
