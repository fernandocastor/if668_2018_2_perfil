\documentclass{article}
\usepackage[utf8]{inputenc}

\title{IF678 - Infraestrutura de Comunicação}
\author{Guilherme Fernandes Xavier da Silva}
\date{October 2018}

\usepackage{natbib}
\usepackage{graphicx}
\usepackage[brazil]{babel}
\usepackage[utf8]{inputenc}
\usepackage[T1]{fontenc}

\begin{document}

\maketitle

\section{Introdução}
A cadeira de infraestrutura de comunicação faz parte da tríade hardware, software e comunicação. Esta faz com que o aluno passe a entender as bases comunicações entre computadores, redes, bem como seus protocolos e aplicações. Além disso, aprende sobre tipos de transporte de dados, história da rede e o surgimento da internet.\cite{cinwiki}

\begin{figure}[h!]
\centering
\includegraphics[scale=1]{Redes_inf.jpg}
\caption{Rede de Computadores\cite{redes}}
\label{fig:redes}
\end{figure}
\section{Relevância}
Como praticamente qualquer sistema de computação atual utiliza de redes como base de construção, e a rede mundial de computadores está tão difundida no dia a dia do profissional de TI e o usuário comum, é de fundamental importância que um aluno de computação entenda seu funcionamento.\cite{cinwiki} 

\section{Relação com outras disciplinas}
% ######## init table ########
\begin{table}[h]
 \centering
% distancia entre a linha e o texto
 {\renewcommand\arraystretch{1.25}
 \caption{ }
 \begin{tabular}{ l l }
  \cline{1-1}\cline{2-2}  
    \multicolumn{1}{|p{3.850cm}|}{Cadeira \centering } &
    \multicolumn{1}{p{4.417cm}|}{Relação \centering }
  \\  
  \cline{1-1}\cline{2-2}  
    \multicolumn{1}{|p{3.850cm}|}{IF741-Gerenciamento de Redes} &
    \multicolumn{1}{p{4.417cm}|}{Em infra-estrutura de comunicação é ensinado a como funcionam as redes, o que é essencial para gerenciamento de redes.}
  \\  
  \cline{1-1}\cline{2-2}  
    \multicolumn{1}{|p{3.850cm}|}{IF712-Programação para Internet} &
    \multicolumn{1}{p{4.417cm}|}{A internet e como ela funciona é um dos principais tópicos de infra-estrutura de comunicação.}
  \\  
  \cline{1-1}\cline{2-2}  
    \multicolumn{1}{|p{3.850cm}|}{IF740-Sistema de Comunicação} &
    \multicolumn{1}{p{4.417cm}|}{Para entender de Sistemas de Comunicação, é necessário ter conhecimentos básicos de infra-estrutura de comunicação.}
  \\  
  \hline

 \end{tabular} }
\end{table}
\section{Bibliografia}
\begin{enumerate}
    \item Redes de Computadores e a Internet - Uma Abordagem Top-Down 5a Edição\cite{redeseinternet5}
    \item Redes de Computadores e a Internet - Uma Abordagem Top-Down 6a edição\cite{redeseinternet6}
    \item Redes de Computadores\cite{redesedecomputadores}
\end{enumerate}




\bibliographystyle{plain}
\bibliography{references}
\end{document}
