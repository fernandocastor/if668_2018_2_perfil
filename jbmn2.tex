\documentclass[10pt]{article}
\usepackage{hyperref}
\usepackage[utf8]{inputenc}
\usepackage[brazil]{babel}

\title{IF668 - Introdução a Computação}
\author{José Bezerra}
\date{Outubro 2018}

\usepackage{natbib}
\usepackage{graphicx}

\begin{document}

\maketitle

\begin{figure}[h!]
\centering
\includegraphics[scale=.16]{ML.png}
\caption{Cérebro Humano Representado com Circuitos}
\label{fig:universe}
\end{figure}

\section{Introdução}
A tecnologia de aprendizagem de máquina é considerada um subcampo da Inteligência Artificial, que trabalha com a ideia de que as máquinas podem aprender sozinhas ao terem acesso a grandes volumes de dados.\citep{DI-UFPE}

A definição mais simples é que as máquinas podem detectar padrões e criar conexões entre dados, por meio de dados e algoritmos sofisticados, para aprenderem sozinhas a executar uma tarefa.\citep{NEWS}

Dentre as diversas modalidades de uso da técnica de Aprendizado de Máquina, a técnica de \textit{deep learning} é a que está em mais relevância no estudo desse campo, que consiste em um avançado algoritmo de detecção de padrões em dados com o fim de executar algum tipo de predição baseada nos dados e padrões que seriam dificilmente detectados por humanos.

Dentro desse campo, a principal ferramenta usada são algoritmos chamados de \textbf{Redes Neurais Artificiais}. Onde a ferramenta será ajustada para o entendimento dos dados e procura de padrões, para que o algoritmo retorne os dados esperados.

\section{Relevância}
Pode-se dizer que aprendizado de máquina seja a tecnologia mais esperada do futuro. Previsões que abrangem de robôs que tomam por si só, cidades inteligentes, telemarketing  a serviços governamentais automatizados (policia, bombeiro etc.) requerem uso dos conhecimentos deste campo tão interessante.

Pode-se afirmar também, que boa parte das empresas de tecnologia do mundo hoje utilizam pelo menos um pouco ou muito de machine learning. 

Não obstante do mercado, a área de inteligência artificial é de vital importância na sociedade moderna, sendo ela possivelmente a responsável pelas próximas grande descobertas da humanidade, seja na maneira como o ser humano interage com serviços em sociedade, a descobertas astronômicas do descobrimento de planetas que possivelmente possam abrigar a raça humana.
\citep{NASA}

\section{Relação com outras disciplinas}
\begin{table}[h!]
\begin{tabular}{|l|l|}
\hline
\textbf{CADEIRA} & \textbf{Relação} \\ \hline
\begin{tabular}[c]{@{}l@{}}IF704 -\\ PROCESSAMENTO \\ LINGUAGEM \\ NATURAL\end{tabular} & \begin{tabular}[c]{@{}l@{}}O Processamento de Linguagem Natural (PLN) \\ é a subárea da Inteligência Artificial (IA) que\\  estuda a capacidade e as limitações de uma \\ máquina em entender a linguagem dos seres humanos\\ com o extensivo uso de redes neurais e técnicas\\  de aprendizado de máquina.\end{tabular} \\ \hline
\begin{tabular}[c]{@{}l@{}}IF702 -\\ REDES NEURAIS\end{tabular} & \begin{tabular}[c]{@{}l@{}}A principal ferramenta utilizada no aprendizado de\\ máquina são algoritmos de redes neurais. Onde nessa\\ disciplina estudaremos redes neurais artificias a fundo\\ para que possamos utilizar em projetos de aprendizado\\ de máquina\end{tabular} \\ \hline
\end{tabular}
\end{table}

\bibliographystyle{plain}
\bibliography{references}
\end{document}
