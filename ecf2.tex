\documentclass[a4paper]{article}
\usepackage[utf8]{inputenc}
\usepackage[brazil]{babel}
\usepackage[T1]{fontenc}

%% Sets page size and margins
\usepackage[a4paper,top=3cm,bottom=2cm,left=3cm,right=3cm,marginparwidth=1.75cm]{geometry}


\title{IF702 - Redes Neurais}
\author{Elaine Cruz Farias}
\date{October 2018}

\usepackage{natbib}
\usepackage{graphicx}

\begin{document}

\maketitle

\section{Introdução}
Redes Neurais é uma disciplina eletiva que geralmente é oferecida pelo professor Germano Vasconcelos no segundo semestre de cada ano. Ela procura abordar os problemas de uma forma diferente som soluções que se baseiam no comportamento do cérebro e dos neurônios para o desenvolvimento de sistemas. Essa disciplina permite ao aluno investigar de forma prática a solução de um problema em uma aplicação de interesse utilizando-se redes neurais. Ela abrange processamento de imagens, como também mineração de dados e mineração na web. Trabalhando habilidades como reconhecimento de padrões e previsão de séries, ela torna-se trivial para o desenvolvimento da aprendizagem de máquina e da inteligência artificial, grande área de computação na qual ela se insere.As redes neurais são sistemas compostos por vários nós que se interconectam em diversas ramificações e, por meio da atualização e ampliação desses laços, bem como interconexões entre eles(que estão associadas a pesos que armazenam o conhecimento da rede), a máquina, pode chegar a conclusões inteligentes e definir padrões, ou seja, aprender.



\section{Relevância}
Redes Neurais é uma cadeira eletiva que prepara o aluno para o atual domínio existente no mercado e na mídia da inteligência artificial e da aprendizagem de máquina. Para os interessados em se engajar nesse setor, os tópicos abordados em tal disciplina são grandes pilares para o ramo.

Apesar de ser introdutória, é interessante agregá-la ao currículo pois seu conhecimento pode ser usado no reconhecimento visual e de fala, no aprendizado de estratégias para controlar robôs, na predição financeira e na bioinformática, possibilitando, pois, uma perspectiva diferenciada na resolução de problemas. 

\subsection{Pontos positivos:}
\begin{itemize}
   \item É uma matéria boa para aprender os fundamentos da área de redes neurais.
    \item A cadeira possui projetos que são bons para exercitar e por em prática os conhecimentos
    \item É uma abordagem alternativa à forma algorítmica de resolver problemas: A partir de exemplos do problema.
    \item Possui uma performance superior à maioria dos algoritmos de aprendizagem de máquina. 
    \item Capacidade de estabelecer relações entre grande quantidade de variáveis.
    \item Enquanto a maioria dos algoritmos de aprendizado atingem um certo limite de eficiência em relação à quantidade de dados fornecida, as redes neurais tendem a se tornarem mais eficientes.
    \item Proporciona uma abordagem simbólica: Toma como base a representação do mundo através de símbolos que representam conceitos.
    
\end{itemize}

\pagebreak

\subsection{Pontos negativos:}
\begin{itemize}
    \item É uma abordagem de problemas que geralmente requer uma grande quantidade de tempo para treinar a rede.
    \item Depende do número de pesos, número de exemplos, de treinamento e outros parâmetros.
    \item Em algumas situações pode-se encontrar dificuldade em entender como e porquê a rede neural gerou um certo resultado.
    \item Geralmente necessitam de mais dados do que outros algoritmos tradicionais da aprendizagem de máquina.
\end{itemize}


\section{Relação com outras disciplinas}

\begin{table}[h]
\begin{tabular}{|c|l|}
\hline
\textbf{Disciplina}                                                                            & \multicolumn{1}{c|}{\textbf{Relação com Redes Neurais}}                                                                                                                                                                                                                                                                                                                                                                                                             \\ \hline
\begin{tabular}[c]{@{}c@{}}IF706 — Introdução a Ciência\\               dos Dados\end{tabular} & \begin{tabular}[c]{@{}l@{}}Esta disciplina objetiva familiarizar o aluno com o \\ novo paradigma científico centrado em dados. Serão\\ apresentadas e discutidas técnicas para integração,\\ visualização, pré-processamento e análise de dados, \\ e comunicação de resultados.Tais técnicas encontram-se \\ profundamente relacionadas com a abordagem de Redes \\ Neurais, que mostra formas de conectar dados e fazer \\ previsões a partir disso.\end{tabular} \\ \hline
\multicolumn{1}{|l|}{IF699 - Aprendizagem de Máquina}                                          & \begin{tabular}[c]{@{}l@{}}Esta disciplina tem como objetivo estudar métodos\\ e algoritmos que obtém conhecimento a partir da \\ análise de bases de dados. A cadeira Redes Neurais \\ trata de alguns métodos que podem ser utilizados, \\ reconhecendo-se, assim, padrões, e fazendo a máquina, \\ de certa forma, aprender.\end{tabular}                                                                                                                        \\ \hline
IF684 - Sistemas Inteligentes                                                                  & \begin{tabular}[c]{@{}l@{}}É uma disciplina obrigatória que faz a introdução de \\ inteligência artificial para  os alunos. Ela cita diferentes \\ tipos de técnicas, algoritmos e áreas dentro de \\ inteligência artificial, incluindo Redes Neurais. Dessa \\ forma, ela pode despertar o  interesse do aluno para \\ aprofundar melhor o assunto. \end{tabular}                                                                                             \\ \hline
\end{tabular}
\caption{Tabela 1: Relação entre Redes Neurais e outras cadeiras.}
\end{table}

\section{Referências}
\begin{itemize}
    \item Neural Computing : An Introduction (Beale, R. e Jackson, T., 1990) \cite{beale1990}
    \item Redes Neurais Artificiais: Teoria e Aplicações (Braga, A.P., Ludermir, T.B, 2000) \cite{braga2000}
    \item Neural Computation: A Comprehensive Foundation (Haykin, S., 1993) \cite{haykin1993}
\end{itemize}

\bibliographystyle{plain}
\bibliography{references}
\end{document}
