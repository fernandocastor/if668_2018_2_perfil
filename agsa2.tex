\documentclass[10pt]{article}
\usepackage[utf8]{inputenc}
\usepackage{float}
\usepackage{graphicx}
\usepackage[portuguese]{babel}
\usepackage{natbib}


\usepackage{geometry}
\geometry{a4paper,left=3cm,top=3cm,right=3cm,bottom=3cm,marginparwidth=1.75cm}

\usepackage[brazil]{babel}
\usepackage[utf8]{inputenc}
\usepackage[TI]{fontenc}

\title{If688 - Teoria e Implementação de Linguagens de Programação}
\author{Alvaro Andrade }
\date{Outubro 2018}


\begin{document}

\maketitle

\section{Introdução}
A matéria de Teoria e  Implementação de Linguagens de Programação ou "Compiladores" como é conhecida por ser equivalente as disciplinas de "if-120 - Compiladores" e "if-473 - Compiladores". Essa cadeira é oferecida no 7º período de ciência da computação pelo professor Leopoldo Motta Teixeira. Seu estudo explora os princípios da logica para computação e desenvolvimento de dados e seu comportamento nos compiladores. O objetivo da cadeira é fornecer os fundamentos para o desenvolvimento da compreensão de teoria de linguagem, análise léxica, representação e geração de códigos e o entendimento da teoria de compiladores. 
 
\begin{figure}[h!]
\centering
\includegraphics[width=0.5\textwidth]{compiler.png}
\caption{Funcionamento de um compilador}
\end{figure}
% imagem 1 : Domínio público  
% https://commons.m.wikimedia.org/wiki/File:Java-program-execution.png

\section{Relevância}
O estudo de compiladores proporciona um entendimento de como são desenvolvidos as ferramentas de um softwares permitindo que o conhecimento do programador vá além da teoria, compreendendo a fundo o funcionamento de uma das ferramentas fundamentais para o funcionamento de um programa.

\subsection{Ponto positivos}
\begin{itemize}
\item Faz com que o aluno de ciência da computação tenha um entendimento completo sobre o funcionamento de um software.
\item Aumenta a compreensão logica do estudante sobre linguagens diversas. 
\end{itemize}

\subsection{Pontos negativos}
\begin{itemize}
\item Alguém que não tenha um bom conhecimento prévio de linguagens de programação tera dificuldades na disciplina. 
\item A cadeira não introduz a teoria básica de logica computacional.
\end{itemize}


\section{Relação com outas disciplinas }

\begin{table}[h]
\centering 
\begin{tabular}{|p{7.0cm}|p{7.0cm}|}
\hline
If689 - Informática teórica & Por abordar temas como as primeiras maquinas a funcionar com programação por software, como a Máquina de Turing. \\ \hline
If669 - Introdução a programação & Essa matéria introduz o aluno a logica de programação e mexe com programas que utiliza compiladores para a execução. \\ \hline
\end{tabular}
\caption{Matérias relacionadas}
\end{table}


\bibliographystyle{plain}
\bibliography{references}
\citep{1, 2, 3, 4, 5}

\end{document}

