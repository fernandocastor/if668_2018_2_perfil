\documentclass{article}

\usepackage[<options>]{natbib}
\renewcommand{\bibsection}{}

\usepackage[utf8]{inputenc}

\title{IF673 - Lógica para Computação}
\author{Jennifer Cristine Batista da Silva}
%\date{October 2018}


\usepackage[brazil]{babel}
\usepackage{natbib}
\usepackage{graphicx}

\begin{document}

\maketitle

\section{Introdução}
A disciplina aborda, de forma introdutória, às técnicas do chamado raciocínio dedutivo usando as ferramentas da Lógica Matemática, que estuda as noções de validade e consistência de argumentos utilizando elementos da Matemática, tais como a teoria dos conjuntos e a álgebra booleana. Os principais tópicos cobertos são: Estruturas Matemáticas e Lógica Matemática (Lógica Proposicional e Lógica de Predicados). Essa disciplina se insere na área de Teoria da Computação, uma das grandes áreas da computação.

\begin{figure}[h!]
\centering
\includegraphics[scale=0.8]{ari}
\caption{Aristóteles, autor do primeiro trabalho sobre lógica}
\label{fig:ari}
\end{figure}

\section{Relevância}
Essa disciplina é importante para que o profissional de Ciência da Computação tenha um raciocínio logicamente estruturado, assim como para aprender a definir logicamente qualquer linguagem bem formulada, viabilizando a leitura correta dos algorítmos, e também promovendo o conhecimento para solução dos problemas que serão apresentados aos profissionais da área. Além do mais, são estudados o conceito de máquina de processamento simbólico e as noções de representação e manipulação simbólica. É uma disciplina fundamental para o entendimento do sistema computacional de um modo geral, dos circuítos de chaveamento dos bits até a lógica das linguagens de alto nível.

\section{Relação com outras disciplinas}

\begin{table}[h]
 \centering

 {\renewcommand\arraystretch{1.25}
 \begin{tabular}{ l l }
  \cline{1-1}\cline{2-2}  
    \multicolumn{1}{|p{4.083cm}|}{IF670 - Matemática Discreta para Computação \centering } &
    \multicolumn{1}{p{4.033cm}|}{Recursão, Indução, Lógica proposicional, Lógica de primeira ordem \centering }
  \\  
  \cline{1-1}\cline{2-2}  
    \multicolumn{1}{|p{4.083cm}|}{IF689 - Informática Teórica \centering } &
    \multicolumn{1}{p{4.033cm}|}{O problema SAT \centering }
  \\  
  \cline{1-1}\cline{2-2}  
    \multicolumn{1}{|p{4.083cm}|}{IF682 - Engenharia de Software e Sistemas \centering } &
    \multicolumn{1}{p{4.033cm}|}{Algoritmos de programação \centering }
  \\  
  \cline{1-1}\cline{2-2}  
    \multicolumn{1}{|p{4.083cm}|}{IF684 - Sistemas Inteligentes \centering } &
    \multicolumn{1}{p{4.033cm}|}{Árvores de decisão \centering }
  \\  
  \cline{1-1}\cline{2-2}  
    \multicolumn{1}{|p{4.083cm}|}{IF773 - Lógicas não Clássica \centering } &
    \multicolumn{1}{p{4.033cm}|}{Sistemas que diferem do padrão, como lógica proposicional e de predicados \centering }
  \\  
  \hline

 \end{tabular} }
\end{table}

\section{Referências}
\bibliographystyle{chicago}
 \cite{dirk2004vandalen}
 \cite{wilfrid1997}
 \cite{jon2000john}
 \cite{jean1986galier}
 \cite{girard1989lafont}
\bibliography{jcbs2}

\end{document}
